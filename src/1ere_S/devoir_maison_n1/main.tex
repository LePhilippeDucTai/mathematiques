\documentclass[12pt,%
addpoints,%
%answers,%
fleqn
]{exam}
\usepackage{pgf}
\usepackage[T1]{fontenc}
\usepackage{tikz,tkz-tab}
\usepackage{pgfplots}
\usetikzlibrary{arrows}
\usepackage{mathrsfs}
\usetikzlibrary{arrows}
\usepackage[utf8]{inputenc}
\usepackage[frenchb]{babel}
\usepackage{amsmath}
\usepackage{amssymb}
\usepackage[left=1.5cm, right=2cm,top=2.5cm,bottom=2.5cm]{geometry}
\usepackage{numprint}
\usepackage{enumerate}
\usepackage{multicol}
\pagestyle{headandfoot}
\firstpageheader{}
  {}
  {Classe : 112 \\ Donné le 4 Janvier 2015 \textbf{pour le 15 Janvier 2016}}
\runningfootrule
\firstpagefootrule
\firstpagefooter{\bsc{Lê} Philippe}{Devoir maison \no 1}{Page \thepage\ sur \numpages}

\runningfooter{\bsc{Lê} Philippe}{Devoir maison \no 1}{Page \thepage\ sur \numpages}

\newcommand{\llbrace}{\left\lbrace}
\newcommand{\rrbrace}{\right\rbrace}



\begin{document}
\pgfplotsset{compat=1.18}


\begin{center}
    {\huge\bfseries Devoir maison \no 1}
    \par\bigskip
\end{center}
\qformat{\textbf{Exercice \thequestion}\quad \thepoints \hfill}

%\boxed{\textbf{Tous les résultats seront encadrés.}} 
\begin{questions}

    \question[5] Soit $a,b$ et $c$ les côtés d'un triangle quelconque. Déterminez l'aire du triangle en fonction de $a,b$ et $c$ en détaillant la démarche.

    \begin{solution}


    \end{solution}


    \question[5] La courbe ci-dessous est la représentation graphique de la fonction $f$ définie sur $[-1;3]$ par
    \[
        f(x) = -x^3 + 3x^2.
    \]
    \begin{parts}
        \part Sans aucune justification, dresser le tableau de variation de $f$.
        \part Déduisez-en le tableau de variation de chacune des fonctions $g$, $h$ et $k$ définies par : \begin{align*}
            g(x) & =f(x)-3,                              \\
            h(x) & =-2f(x),                              \\
            k(x) & =\dfrac{1}{f(x)},                     \\
            l(x) & =\dfrac{1}{\sqrt{-\dfrac{f(x)}{2}+2}}
        \end{align*}
        Justifiez le sens de variation des fonctions $g,h,k,$ et $l$. Attention aux valeurs interdites !

        \begin{tikzpicture}
            \begin{axis}[
                    grid=both,
                    axis lines = center,
                    xlabel = $x$,
                    ylabel = {$f(x)$},
                    xtick={-1,0,1,2,3}
                ]
                %Below the red parabola is defined
                \addplot [
                    domain=-1:3,
                    samples=200,
                    color=black,
                ]
                {-x^3+3*x^2};

            \end{axis}
        \end{tikzpicture}

    \end{parts}

    \question[10] Le but de cet exercice sera de déterminer l'ensemble de définition ainsi que le signe de la fonction $f$ définie par :
    \[
        f:x\mapsto \dfrac{1}{\sqrt{-x^2+x+6}-\sqrt{2x^2-3x+1}}.
    \]
    On pose les fonctions suivantes :
    \begin{align*}
        P_1(x) & = -x^2 + x + 6,  \\
        P_2(x) & = 2x^2 - 3x + 1, \\
        h(x)   & = \sqrt{P_1(x)}, \\
        k(x)   & = \sqrt{P_2(x)}, \\
        g(x)   & = h(x)-k(x).
    \end{align*}

    \begin{parts}
        \part On souhaite tout d'abord déterminer l'ensemble de définition de $g$ (qui n'est autre que le dénominateur de $f$). Pour cela, voici des questions préliminaires :
        \begin{subparts}
            \subpart Dresser le tableau de signes de $P_1$ et en déduire l'ensemble de définition de $h$, noté $D_h$.
            \subpart Dresser le tableau de signes de $P_2$ et en déduire l'ensemble de définition de $k$, noté $D_k$.
            \subpart En déduire l'ensemble de définition de $g$, noté $D_g$. Pourquoi $D_g = D_h \cap D_k$ ?
        \end{subparts}
        \part \`A présent, on souhaite voir pour quelles valeurs de $g(x)$, la fonction $f(x)$ est bien définie.
        \begin{subparts}
            \subpart Dresser le tableau de signes de $g$ sur $D_g$.
            \subpart En déduire les valeurs interdites que $f$ ne peut prendre.
            \subpart En déduire l'ensemble de définition de $f$, noté $D_f$.
        \end{subparts}

        \part Dresser le tableau de signes de $f$ sur $D_f$.

    \end{parts}


    \question[2] (Bonus)
    Donnez le complémentaire de la phrase $P$ : \og La nuit, tous les chats sont gris \fg.\\
    On cherche donc une phrase $Q$ telle que si $Q$ est vraie, alors la phrase $P$ sera fausse.




\end{questions}

\end{document}


