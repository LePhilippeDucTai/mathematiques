\documentclass[12pt,
addpoints,
answers,
fleqn
]{exam}
\usepackage{fix-cm}
\usepackage[T1]{fontenc}
\usepackage{pgf}
\usepackage{tikz,tkz-tab}
\usetikzlibrary{arrows}
\usepackage{mathrsfs}
\usetikzlibrary{arrows}
\usepackage[utf8]{inputenc}
\usepackage[frenchb]{babel}
\usepackage{amsmath}
\usepackage{amssymb}
\usepackage[left=1.5cm, right=2cm,top=2.5cm,bottom=2.5cm]{geometry}
\usepackage{numprint}
\usepackage{enumerate}
\usepackage{graphicx}
\usepackage{multicol}
\usepackage{cancel}
\pagestyle{headandfoot}
\firstpageheader{}
  {}
  {Classe : 217 \\ Donné le 7 Décembre 2015 \textbf{pour le 4 Janvier 2016}}
\runningfootrule
\firstpagefootrule
\firstpagefooter{\bsc{Lê} Philippe}{Devoir maison \no 1}{Page \thepage\ sur \numpages}

\runningfooter{\bsc{Lê} Philippe}{Devoir maison \no 1}{Page \thepage\ sur \numpages}

\newcommand{\llbrace}{\left\lbrace}
\newcommand{\rrbrace}{\right\rbrace}



\begin{document}



\begin{center}
    {\huge\bfseries Devoir maison \no 1}
    \par\bigskip
\end{center}
\qformat{\textbf{Exercice \thequestion}\quad \thepoints \hfill}

\boxed{\textbf{Tous les résultats seront encadrés.}}
\begin{questions}

    \question (Calcul numérique) Calculer en donnant les résultats sous la forme d'une fraction irréductible. Détaillez les calculs.
    \begin{multicols}{3}
        \begin{itemize}
            \item[] $a= \dfrac{2+\frac{2}{3}}{3-\frac{1}{2}}$ \\
            \item[] $b=\left( \dfrac{3}{2}+2\right)\times \dfrac{2}{5}$ \\
            \item[] $c=\dfrac{\frac{2}{3}+\frac{1}{2}}{\frac{5}{4}-4}$ \\
            \item[] $d=\dfrac{2+\frac{1}{3}}{\frac{1}{2}-5\times\frac{2}{3}}$ \\
            \item[] $ e=\left(2 - \dfrac{2}{3}\right)\left(\dfrac{5}{2}\times 3 - \dfrac{2}{3}\right)$
            \item[] $f=\dfrac{1}{\frac{2}{3}}\times\dfrac{\frac{3}{4}}{5}.$
        \end{itemize}
    \end{multicols}

    \begin{solution}
        \begin{align*}
            a & = \dfrac{2+\frac{2}{3}}{3-\frac{1}{2}} = \dfrac{\frac{6+2}{3}}{\frac{6-1}{2}} = \dfrac{\frac{8}{3}}{\frac{5}{2}} = \dfrac{8}{3}\times\dfrac{2}{5} = \boxed{\dfrac{16}{15}}.                                                                                                                                \\
            b & =\left(\dfrac{3}{2}+2\right)\times \dfrac{2}{5} = \left(\dfrac{3}{2}+\dfrac{4}{2}\right)\times\dfrac{2}{5} = \dfrac{7}{2}\times \dfrac{2}{5} = \boxed{\dfrac{7}{5}}.                                                                                                                                       \\
            c & =\dfrac{\frac{2}{3}+\frac{1}{2}}{\frac{5}{4}-4} = \dfrac{\frac{4}{6}+\frac{3}{6}}{\frac{5}{4}-\frac{16}{4}}=\dfrac{\frac{7}{6}}{-\frac{11}{4}}=\dfrac{7}{6}\times\dfrac{-4}{11}=\dfrac{7}{3}\times\dfrac{-2}{11} = \boxed{-\dfrac{14}{33}}.                                                                \\
            d & =\dfrac{2+\frac{1}{3}}{\frac{1}{2}-5\times\frac{2}{3}} = \dfrac{\frac{6+1}{3}}{\frac{1}{2}-\frac{10}{3}}=\dfrac{\frac{7}{3}}{\frac{3-20}{6}}=\dfrac{\frac{7}{3}}{\frac{-17}{6}}=\dfrac{7}{3}\times \dfrac{6}{-17} = 7\times\dfrac{2}{-17}=\boxed{-\dfrac{14}{17}}.                                         \\
            e & =\left(2-\dfrac{2}{3}\right)\left(\dfrac{5}{2}\times 3 - \dfrac{2}{3}\right) = \left(\dfrac{6-2}{3}\right)\left(\dfrac{15}{2}-\dfrac{2}{3}\right) = \dfrac{4}{3}\times\left( \dfrac{45}{6}-\dfrac{4}{6}\right) = \dfrac{4}{3}\times\dfrac{41}{6} =\dfrac{2}{3}\times\dfrac{41}{3} = \boxed{\dfrac{82}{9}}. \\
            f & =\dfrac{1}{\frac{2}{3}}\times\dfrac{\frac{3}{4}}{5} = \dfrac{3}{2}\times\dfrac{3}{4}\times\dfrac{1}{5}= \boxed{\dfrac{9}{40}}.
        \end{align*}

    \end{solution}


    \question (Simplifications de puissances) Simplifier les expressions suivantes :
    \begin{align*}
        a & = \left(2^3\times 5^2\right)^3\times 2^4 \times 5^2 & b & =\left(\dfrac{3^2}{2^3}\right)^4\times\left(\dfrac{2}{3^3}\right)^2 & c & =16^3\times 27^2                                          \\
        d & =\dfrac{8^3\times 4^2}{32^4}                        & e & =\dfrac{\left(2^4\times 3^2\right)^2}{\left(2^3\times 3\right)^3}   & f & =\dfrac{(x^2y)\left(xy^3\right)^2}{\left(x^3y^2\right)^3}
    \end{align*}

    \begin{solution}
        \begin{align*}
            a & =\left(2^3\times 5^2\right)^3 \times 2^4 \times 5^2 = 2^9\times 5^6 \times 2^4 \times 5^2 = \boxed{2^{13}\times 5^8}.                              \\
            b & =\left(\dfrac{3^2}{2^3}\right)^4\times\left(\dfrac{2}{3^3}\right)^2 = \dfrac{3^8}{2^12}\times\dfrac{2^2}{3^6}=\boxed{\dfrac{3^2}{2^{10}}}.         \\
            c & =16^3\times 27^2 = \left(2^4\right)^3\times\left(3^3\right)^2 = \boxed{2^{12}\times 3^6}.                                                          \\
            d & =\dfrac{8^3\times 4^2}{32^4} = \dfrac{(2^3)^3\times(2^2)^2}{(2^5)^4}= \dfrac{2^9\times 2^4}{2^{20}}=\dfrac{2^{13}}{2^{20}}=\boxed{\dfrac{1}{2^7}}. \\
            e & =\dfrac{\left(2^4\times 3^2\right)^2}{\left(2^3\times 3\right)^3} = \dfrac{2^8\times 3^4}{2^9\times 3^3}=\boxed{\dfrac{3}{2}}.                     \\
            f & =\dfrac{(x^2y)(xy^3)^2}{(x^3y^2)^3}=\dfrac{x^2yx^2y^6}{x^9y^6}=\dfrac{x^4 y^7}{x^9y^6} =\boxed{\dfrac{y}{x^5}}.
        \end{align*}

    \end{solution}

    %\newpage
    \question (Résolution d'équations) Résoudre les équations suivantes :
    \begin{multicols}{3}
        \begin{enumerate}[(a)]
            \item $\dfrac{x}{3}=2+1$
            \item $\dfrac{2x+3}{2}=\dfrac{1}{5}$
            \item $2=\dfrac{3x-\dfrac{1}{2}}{4}$
            \item $\dfrac{5x-1}{3}=2x$
            \item $\dfrac{1-\frac{1}{5}x}{2}=\dfrac{3x}{5}$
            \item $\dfrac{1}{3}+\dfrac{1}{6}\left(2-\dfrac{x}{-2}\right)= -\dfrac{2}{3}x$
        \end{enumerate}
    \end{multicols}

    \begin{solution}
        \begin{parts}
            \part $\dfrac{x}{3}=2+1 \iff \dfrac{x}{3}=3 \iff x=9.$ Alors, $\boxed{S=\left\lbrace 9 \right\rbrace}$.
            \part $\dfrac{2x+3}{2}=\dfrac{1}{5} \iff 2x+3 =\dfrac{2}{5} \iff 2x = \dfrac{2}{5}-3 \iff 2x= \dfrac{2-15}{5} \iff 2x= \dfrac{-13}{5} \iff x=\dfrac{-13}{10}$. Alors $\boxed{ S=\left\lbrace -\dfrac{13}{10}\right\rbrace}$.
            \part $ 2 = \dfrac{3x-\dfrac{1}{2}}{4} \iff 8 = 3x - \dfrac{1}{2} \iff 8 +\dfrac{1}{2}= 3x \iff 3x= \dfrac{16+1}{2} \iff x= \dfrac{17}{6}$. Alors, $ \boxed{ S= \llbrace \dfrac{17}{6}\rrbrace}$.

            \part $\dfrac{5x-1}{3}=2x \iff 5x-1 = 6x \iff -1 = 6x-5x \iff x=-1$. Alors $\boxed{S=\llbrace -1 \rrbrace}$.
            \part $\dfrac{1-\frac{1}{5}x}{2} = \dfrac{3x}{5} \iff 1-\dfrac{x}{5} = \dfrac{6x}{5} \iff 1=\dfrac{6x+x}{5} \iff \dfrac{7x}{5}=1 \iff x=\dfrac{5}{7}$. Alors, $\boxed{S=\llbrace \dfrac{5}{7}\rrbrace}$.

            \part $\dfrac{1}{3}+\dfrac{1}{6}\left(2-\dfrac{x}{-2}\right)= -\dfrac{2}{3}x \iff \dfrac{1}{3}+\dfrac{1}{6}\left(2+\dfrac{x}{2}\right)=-\dfrac{2}{3}x \iff \dfrac{1}{3}+\dfrac{1}{6}\left(\dfrac{4+x}{2}\right) = -\dfrac{2x}{3} \iff \dfrac{1}{3}+ \dfrac{4+x}{12} = -\dfrac{2x}{3} \iff \dfrac{1}{3}+\dfrac{4}{12}=\dfrac{-2x}{3}-\dfrac{x}{12} \iff \dfrac{2}{3} = \dfrac{-8x-x}{12} \iff \dfrac{2}{3}=\dfrac{-9x}{12} \iff x= \dfrac{2}{3}\times \dfrac{12}{-9} = -2\times\dfrac{4}{9} \iff x=-\dfrac{8}{9}$. Alors, $\boxed{S=\llbrace -\dfrac{8}{9}\rrbrace}$.
        \end{parts}
    \end{solution}

    \question (Résolution d'équations) Résoudre les équations suivantes :
    \begin{multicols}{2}
        \begin{enumerate}[(1)]
            \item $(2x-1)^2 + (x-1)(2x-1) = 0$
            \item $(x-1)(2x-2) + (2x-2)(5-2x) = 0$
            \item $(3x-1)(2x+2)+3(5-2x)(x+1)= 0$
            \item $(4x+6)(1-2x)=5(2x+3)^2$
            \item $\dfrac{x-3}{x+2}-\dfrac{-x-2}{3x+1}=0$
            \item $\dfrac{2x}{4x+1}=\dfrac{x+1}{2x-1}$
        \end{enumerate}
    \end{multicols}

    \begin{solution}
        \begin{parts}
            \part $(2x-1)^2 + (x-1)(2x-1) = 0 \iff (2x-1)\left[(2x-1)+(x-1)\right]= 0 \iff (2x-1)(3x-2) = 0 \iff 2x-1 = 0 \text{ ou } 3x-2= 0 \iff x=\dfrac{1}{2} \text{ ou } x=\dfrac{2}{3}$.
            Donc, l'ensemble des solutions est $\boxed{S=\llbrace \dfrac{1}{2}; \dfrac{2}{3}\rrbrace}$.

            \part $(x-1)(2x-2) + (2x-2)(5-2x) = 0 \iff (2x-2)\left[ (x-1)+(5-2x)\right] = 0 \iff (2x-2)(x-1+5-2x) = 0 \iff (2x-2)(4-x) =0 \iff 2x-2=0 \text{ ou } 4-x=0 \iff x=1 \text{ ou } x=4$. Donc l'ensemble des solutions est $ \boxed{S = \llbrace 1 ; 4 \rrbrace}$.

            \part $(3x-1)(2x+2)+3(5-2x)(x+1)= 0 \iff (3x-1)2(x+1)+3(5-2x)(x+1) = 0 \iff (x+1)\left[2(3x-1)+3(5-2x)\right] = 0 \iff (x+1)(6x-2+15-6x) = 0 \iff (x+1)(13)= 0 \iff x+1= 0$. Donc l'ensemble des solutions est $\boxed{S=\llbrace -1\rrbrace}$.

            \part $(4x+6)(1-2x)=5(2x+3)^2 \iff 2(2x+3)(1-2x)-5(2x+3)^2=0 \iff (2x+3)\left[2(1-2x)-5(2x+3)\right] =0 \iff (2x+3)(2-4x-10x-15)=0 \iff (2x+3)(-14x-13)=0 \iff 2x+3 = 0 \text{ ou } -14x-13 = 0 \iff x=-\dfrac{3}{2} \text{ ou } x=-\dfrac{13}{14}$. Donc, $\boxed{S = \llbrace -\dfrac{-3}{2} ; \dfrac{-13}{14}\rrbrace}$.

            \part Valeurs interdites : $2$ et $-\dfrac{1}{3}$.\\ $\dfrac{x-3}{x+2}-\dfrac{-x-2}{3x+1}=0$\\
            $\iff \dfrac{(x-3)(3x+1)-(-x-2)(x+2)}{(x+2)(3x+1)}= 0 \iff
                3x^2+x-9x-3-(-x^2-2x-2x-4) = 0 \iff 3x^2-8x-3+x^2+4x+4=0 \iff 4x^2-4x+1= 0 \iff (2x-1)^2=0 \iff 2x-1=0 \iff x=\dfrac{1}{2}$. Alors $\boxed{S=\llbrace \dfrac{1}{2}\rrbrace}$.

            \part Valeurs interdites : $-\dfrac{1}{4}$, $\dfrac{1}{2}$.\\
            $\dfrac{2x}{4x+1}=\dfrac{x+1}{2x-1} \iff \dfrac{2x}{4x+1}-\dfrac{x+1}{2x-1}=0$ \\
            $\iff \dfrac{2x(2x-1)-(x+1)(4x+1)}{(4x+1)(2x-1)}= 0 \iff 4x^2-2x- (4x^2+x+4x+1) = 0 \iff 4x^2-2x-4x^2-x-4x-1 = 0 \iff -7x-1=0 \iff x=-\dfrac{1}{7}$. Alors $\boxed{S=\llbrace -\dfrac{1}{7}\rrbrace}$.
        \end{parts}

    \end{solution}

    \question (Problème d'optimisation) Un agriculteur dispose de 200 mètres de clôture. \`A l'aide de cette clôture, il souhaite entourer la plus grande partie de son champ ;  cette partie doit avoir une forme rectangulaire. On note $x$ et $y$ la longueur et la largeur respectives de cette partie rectangulaire. On note l'aire de cette partie clôturée, la fonction à deux variables $\mathcal{A}$ :
    \[ \mathcal{A}(x,y) = xy \]
    \begin{parts}
        \part Montrer que :
        \[ \mathcal{A}(x,y) =\dfrac{1}{4}(x+y)^2 - \dfrac{1}{4}(x-y)^2.
        \]
        \part Justifier pourquoi $2x+2y=200$.
        \part Montrer que \textbf{sous la contrainte} $x+y=100$, maximiser $\mathcal{A}$ revient à maximiser la fonction \[f(t)= 50^2 - \dfrac{1}{4}\left(2t-100\right)^2,\]
        pour $t \in [0;100]$.
        \part Justifier pour quel $t \in [0;100] $ la fonction $f$ est maximale. En déduire les dimensions de $x$ et $y$ afin que l'aire de la partie clôturée soit maximale. Donner l'aire du champ obtenu en $m^2$.
    \end{parts}

    \begin{solution}
        \begin{parts}
            \part Il suffit de développer le terme de droite et obtenir que cela vaut $xy$.
            \begin{align*}
                \dfrac{1}{4}(x+y)^2 - \dfrac{1}{4}(x-y)^2 & = \dfrac{1}{4}\left( x^2 + 2xy+ y^2\right) - \dfrac{1}{4}\left(x^2-2xy+y^2\right) \\
                                                          & = \dfrac{1}{4}\left(x^2+2xy+y^2 - x^2 +2xy - y^2\right)                           \\
                                                          & = \dfrac{1}{4}\left(4xy\right)                                                    \\
                                                          & = xy = \mathcal{A}(x,y).
            \end{align*}
            On a bien $\mathcal{A}(x,y)=\dfrac{1}{4}(x+y)^2 - \dfrac{1}{4}(x-y)^2$.

            \part L'agriculteur dispose de $200$ mètres de clôture. Ainsi, le périmètre final sera forcémment de $200$ mètres. La clôture sera de forme rectangulaire, et le périmètre d'un rectangle est égale à $2\times Longeur + 2\times largeur$. Ainsi, $x$ étant la longueur et $y$ étant la largeur, forcémment, $2x+2y=200$.

            \part Dans la question (a), on intègre la contrainte dans l'expression trouvée de $\mathcal{A}(x,y)$. On peut exprimer $y$ en fonction de $x$ par la contrainte :
            \begin{align*}
                x+y    & =100    \\
                \iff y & =100-x.
            \end{align*}
            Alors en remplaçant dans $\mathcal{A}$ de la question (a) :
            \begin{align*}
                \mathcal{A}(x,100-x) & =\dfrac{1}{4}(x+100-x)^2 - \dfrac{1}{4}(x-(100-x))^2                \\
                                     & = \dfrac{1}{4}(100)^2 - \dfrac{1}{4}(x-100+x)^2                     \\
                                     & = \dfrac{1}{4}100^2 - \dfrac{1}{4}(2x-100)^2                        \\
                                     & = \left(\dfrac{100}{2}\right)^2 - \dfrac{1}{4}\left(2x-100\right)^2 \\
                                     & = 50^2 - \dfrac{1}{4}(2x-100)^2.
            \end{align*}

            Alors, maximiser $\mathcal{A}$ sous la contrainte $x+y=100$ revient donc à maximiser la fonction :
            \[ f(t) = 50^2 - \dfrac{1}{4}(2x-100)^2.
            \] sur l'intervalle $[0,100]$.

            \part Pour tout $t \in [0;100]$, $- \dfrac{1}{4}(2t-100)^2 \leq 0 $, car un carré est toujours positif, et multiplié par $-1$ le rend négatif. Ainsi, en ajoutant $50^2$ des deux côtés de l'inégalité on a pour tout $t \in [0;100]$ : $50^2 - \dfrac{1}{4}(2t-100)^2  \leq 50^2$. Soit donc $f(t) \leq 50^2$ pour tout $t\in [0;100]$.
            De plus, ce maximum est atteint lorsque le terme $-\dfrac{1}{4}(2t-100)^2 = 0$. Cela est possible seulement si $(2t-100)^2=0 \iff 2t-100=0 \iff 2t=100 \iff t=50$. Alors, $f(50)=50^2$.

            Ainsi, pour $t=50$, la fonction $f$ admet un maximum. Si $x=50$, alors $y=50$ car on sait que $x+y=100$. Ainsi, la partie clôturée sera carrée et d'aire égale à $50\times 50 =2500 m^2$.

        \end{parts}
    \end{solution}

    \question (Points d'intersection) On considère les deux fonctions $f$ et $g$ dont les images d'un nombre $x$ sont données par les relations :
    \begin{align*}
        f(x) & =-\dfrac{1}{4}x^2 +x +3 \\ g(x)&=\dfrac{x}{2} +1
    \end{align*}
    \begin{parts}
        \part Représenter $\mathcal{C}_f$ et $\mathcal{C}_g$ les courbes représentatives de $f$ et $g$ sur un repère orthonormé.
        \part \begin{subparts} \subpart Déterminer les valeurs des réels $a$ et $b$ vérifiant l'égalité :
            \[ -x^2 +4x +12 = (x-6)(ax+b). \]
            \subpart En déduire les solutions des équations $f(x)=0$.
        \end{subparts}
        \part \begin{subparts} \subpart \'Etablir l'égalité suivante :
            \[ -\dfrac{x^2}{4}+\dfrac{x}{2}+2 = -\dfrac{(x-4)(x+2)}{4}. \]
            \subpart Résoudre l'équation : $f(x) = g(x)$.
            \subpart En déduire les coordonnées des points d'intersection des courbes $\mathcal{C}_f$ et $\mathcal{C}_g$.
        \end{subparts}
    \end{parts}

    \begin{solution}
        \begin{parts}
            \part aa
            \part \begin{subparts}
                \subpart Il suffit dans cette question de développer le terme de droite et identifier le termes associés à $x^2$, $x$ et les constantes avec les constantes :
                \begin{align*}
                    (x-6)(ax+b) & = ax^2+ bx - 6ax - 6b \\
                                & = ax^2 + (b-6)x - 6b.
                \end{align*}
                Cette dernière quantité est égale à $-x^2 + 4x+12$. Ainsi, on identifie par les coefficients :
                \[
                    \left \{
                    \begin{array}{rcr}
                        a   & = & -1 \\
                        b-6 & = & 4  \\
                        -6b & = & 12 \\
                    \end{array}
                    \right.
                    \iff\left\{
                    \begin{array}{rcr}
                        a   & = & -1             \\
                        b+6 & = & 4              \\
                        b   & = & \dfrac{12}{-6} \\
                    \end{array}
                    \right.
                \]
                \[
                    \iff\boxed{\left\{
                        \begin{array}{rcr}
                            a & = & -1 \\
                            b & = & -2 \\
                        \end{array}
                        \right.}
                \]

                D'où, \[ -x^2 +4x+12 = (x-6)(-x-2). \]
                \subpart $f(x)=0 \iff -\dfrac{1}{4}x^2 + x+3 = 0$. On remarque que en multipliant par $4$ des deux côtés de l'égalité, on tombe sur l'équation $-x^2+4x+12=0$. Celle ci est vraie si et seulement si $(x-6)(-x-2)=0 \iff x-6 = 0 \text{ ou } -x-2=0 \iff x=6 \text{ ou } x=-2$. L'ensemble des solutions est donc : $\boxed{S=\llbrace -2 ; 6 \rrbrace}$.
            \end{subparts}

            \part \begin{subparts}
                \subpart Il suffit là aussi de développer le terme de droite et d'obtenir que cela est égal au terme de gauche :
                \begin{align*} -\dfrac{1}{4}(x-4)(x+2) & = -\dfrac{1}{4}(x^2+2x-4x-8)                 \\
                                       & = -\dfrac{1}{4}(x^2-2x-8)                    \\
                                       & = -\dfrac{x^2}{4}+\dfrac{2x}{4}+\dfrac{8}{4} \\
                                       & = \boxed{-\dfrac{x^2}{4}+ \dfrac{x}{2}+2}.
                \end{align*}
                \subpart L'équation $f(x)=g(x)$ se ramène à résoudre :
                \begin{align*}
                    -\dfrac{1}{4}x^2 + x + 3                  & = \dfrac{x}{2}+1 \\
                    \iff -\dfrac{x^2}{4}+x+3 - \dfrac{x}{2}-1 & = 0              \\
                    \iff -\dfrac{x^2}{4}+\dfrac{x}{2}+2       & =0.
                \end{align*}
                On a vu à la question précédente que cette dernière équation était équivalente à $-\dfrac{(x-4)(x+2)}{4}= 0$. Celle-ci est résolue pour $x=4$ et $x=-2$.
                D'où, l'ensemble des solutions de l'équation $f(x)=g(x)$ sont $S=\llbrace -2 ; 4 \rrbrace$.

                \subpart On sait donc que $f(x)=g(x)$ si et seulement si $x=-2$ ou $x=4$. Donc en $x=-2$ et en $x=4$ les courbes $\mathcal{C}_f$ et $\mathcal{C}_g$ se rencontrent en les points $\left(-2;g(-2)\right)$ et $\left(4;g(4)\right)$.
                Il suffit maintenant de calculer les images de $-2$ et $4$ par $f$ :
                \begin{align*}
                    g(-2) & = \dfrac{-2}{2}+1 \\
                          & = -1+1 = 0.       \\
                    g(4)  & = \dfrac{4}{2}+1  \\
                          & = 2+1 = 3.
                \end{align*}
                Alors, les courbes $\mathcal{C}_f$ et $\mathcal{C}_g$ s'intersectent aux points de coordonnées $(-2;0)$, et $(4;3)$.
            \end{subparts}

        \end{parts}
    \end{solution}

    \question (Résolution d'inéquations) Résoudre les inéquations suivantes dans $\mathbb{R}$. Donner les solutions sous forme d'ensemble.
    \begin{multicols}{2}
        \begin{enumerate}[(1)]
            \item $2x \geq  0$
            \item $3x+1 \geq 2$
            \item $3x-3 > 0$
            \item $\dfrac{3}{7}+\dfrac{1}{2}x < 2$
            \item $-\dfrac{5}{2}+\dfrac{\sqrt{2}x}{-3} > 0$
            \item $\dfrac{4\sqrt{8}x+1}{5} > \dfrac{3\sqrt{2}}{4}x$
        \end{enumerate}
    \end{multicols}

    \begin{solution}
        \begin{parts}
            \part $2x \geq 0 \iff x\geq 0$. $\boxed{S=[0;+\infty[}$.
            \part $3x+1 \geq 2 \iff 3x \geq 2 -1 \iff 3x \geq 1 \iff x\geq \dfrac{1}{3}$. Alors $\boxed{S=\left[\dfrac{1}{3};+\infty\right]}$.
            \part $3x-3 > 0 \iff 3x > 3 \iff x > 1$. $\boxed{S=]1;+\infty[}$.
            \part $\dfrac{3}{7}+\dfrac{1}{2}x < 2 \iff \dfrac{x}{2} < 2 - \dfrac{3}{7}\iff \dfrac{x}{2} < \dfrac{14-3}{7} \iff x < \dfrac{11\times 2}{7} \iff x < \dfrac{22}{7}$. Alors $\boxed{S=\left]-\infty; \dfrac{22}{7}\right[}$.
            \part $-\dfrac{5}{2} + \dfrac{\sqrt{2}x}{-3} > 0 \iff \dfrac{\sqrt{2}x}{-3} > \dfrac{5}{2} \iff x < \dfrac{5}{2}\times\dfrac{-3}{\sqrt{2}}$ (On change le sens de l'inégalité lorsqu'on multiplie par $-1$ des deux côtés de l'inégalité)
            $\iff x < \dfrac{-15}{2\sqrt{2}}$. Alors, $\boxed{S=\left]-\infty ; -\dfrac{15}{2\sqrt{2}}\right[}$.

            \part $\dfrac{4\sqrt{8}x+1}{5}> \dfrac{3\sqrt{2}}{4}x \iff \dfrac{4\sqrt{8}x}{5} - \dfrac{3\sqrt{2}x}{4} > \dfrac{-1}{5} \iff \left(\dfrac{4\times 2\sqrt{2}}{5}-\dfrac{3\sqrt{2}}{4}\right)x > \dfrac{-1}{5} \iff \left(\dfrac{32\sqrt{2}}{20}-\dfrac{15\sqrt{2}}{20}\right)x > \dfrac{-1}{5} \iff \dfrac{17\sqrt{2}}{20}x > -\dfrac{1}{5} \iff x > \dfrac{-1}{5}\times\dfrac{20}{17\sqrt{2}} \iff x > \dfrac{-4}{17\sqrt{2}}$. Alors,  $\boxed{S=]-\dfrac{-4}{17\sqrt{2}} ; +\infty[ }$.

        \end{parts}

    \end{solution}


    \question (Résolution d'inéquations) Résoudre les inéquations suivantes dans $\mathbb{R}$ en établissant un tableau de signes. Donner les solutions sous forme d'ensemble.

    \begin{multicols}{2}
        \begin{enumerate}[(1)]
            \item $x^2 > x$
            \item $x^2 - 4 \geq 0$
            \item $(2x+1)(x+3) \geq 0$
            \item $(x+1)^2(2x+3) \leq 0$
            \item $(3-x)(2x+5)(x+3) \geq 0$
            \item $(2x+3)(2-x) + (2-x)(3x-1) \geq 0$
            \item $(2x-1)(2-x)-\left(\dfrac{x}{2}+1\right)(5-4x) \leq 0$
            \item $(x-1)^2-(2x+1)^2 \geq 0$
        \end{enumerate}
    \end{multicols}

    \begin{solution}
        \begin{parts}
            \part $x^2 > x \iff x^2 - x > 0 \iff x(x-1)>0$.\\
            \begin{tikzpicture}
                \tkzTabInit[lgt=4]{$x$/1,
                    $x$/1,
                    $x-1$/1,
                    $x(x-1)$/1}{$-\infty$,$ 0 $,$1$,$+\infty$}
                \tkzTabLine{,-,z,+,t,+,}
                \tkzTabLine{,-,t,-,z,+,}
                \tkzTabLine{,+,z,-,z,+,}
            \end{tikzpicture}\\
            On cherche lorsque le terme $x(x-1)$ est positif. On cherche alors les signes $+$ sur le tableau de signe et on donne l'ensemble associé. Alors, \[\boxed{S =\left] -\infty ; 0\right[\cup\left]1;+\infty\right[}.\]
                On n'oublie pas les crochets ouverts pour les solutions en $0$ car l'inégalité de l'inéquation est \textbf{stricte}.

                \part $x^2 - 4 \geq 0 \iff (x-2)(x+2)\geq 0.$\\
                \begin{tikzpicture}
                    \tkzTabInit[lgt=4]{$x$/1,
                        $x-2$/1,
                        $x+2$/1,
                        $(x-2)(x+2)$/1}{$-\infty$,$ -2 $,$2$,$+\infty$}
                    \tkzTabLine{,-,t,-,z,+,}
                    \tkzTabLine{,-,z,+,t,+,}
                    \tkzTabLine{,+,z,-,z,+,}
                \end{tikzpicture}\\
                On cherche dans l'inéquation lorsque $(x-2)(x+2)$ est positif. D'après le tableau de signes, cela n'est possible que sur l'intervalle $]-\infty ; -2]$ ou $[2 ; +\infty[$. Alors,
                \[ \boxed{S= ]-\infty ; -2] \cup [2;+\infty[}.\]
                Inégalités larges, donc on inclut $-2$ et $2$ dans les intervalles.

                \part $(2x+1)(x+3)\geq 0$.\\
                \begin{tikzpicture}
                    \tkzTabInit[lgt=4]{$x$/1,
                        $2x+1$/1,
                        $x+3$/1,
                        $(2x+1)(x+3)$/1}{$-\infty$,$ -3 $,$-\dfrac{1}{2}$,$+\infty$}
                    \tkzTabLine{,-,t,-,z,+,}
                    \tkzTabLine{,-,z,+,t,+,}
                    \tkzTabLine{,+,z,-,z,+,}
                \end{tikzpicture}\\
                \[\boxed{S= \left[-\infty ; -3\right]\cup\left[-\dfrac{1}{2};+\infty\right[.}
                \]

                \part $(x+1)^2(2x+3) \leq 0$.\\
                \begin{tikzpicture}
                    \tkzTabInit[lgt=4]{$x$/1,
                        $x+1$/1,
                        $x+1$/1,
                        $2x+3$/1,
                        $(x+1)^2(2x+3)$/1}{$-\infty$,$ -\dfrac{3}{2} $,$-1$,$+\infty$}
                    \tkzTabLine{,-,t,-,z,+,}
                    \tkzTabLine{,-,t,-,z,+,}
                    \tkzTabLine{,-,z,+,t,+,}
                    \tkzTabLine{,-,z,+,z,+,}
                \end{tikzpicture}\\
                \[
                    \boxed{S=\left]-\infty; -\dfrac{3}{2}\right]}.
                \]

                \part $(3-x)(2x+5)(x+3) \geq 0$.\\
                \begin{align*}
                     & 3-x \geq 0 \iff x \leq 3.                             \\
                     & 2x+5 \geq 0 \iff 2x\geq -5 \iff x \geq -\dfrac{5}{2}. \\
                     & x+3 \geq 0 \iff x \geq -3.
                \end{align*}

                \begin{tikzpicture}
                    \tkzTabInit[lgt=3]{$x$/1,
                        $3-x$/1,
                        $2x+5$/1,
                        $x+3$/1,
                        $(3-x)(2x+5)(x+3)$/1}{$-\infty$,$ -3 $,$-\dfrac{5}{2}$,$3$,$+\infty$}
                    \tkzTabLine{,+,t,+,t,+,z,-,}
                    \tkzTabLine{,-,t,-,z,+,t,+,}
                    \tkzTabLine{,-,z,+,t,+,t,+,}
                    \tkzTabLine{,+,z,-,z,+,z,-,}
                \end{tikzpicture}\\

                Par conséquent :
                \[
                    \boxed{S = \left]-\infty ; -3\right]\cup\left[-\dfrac{5}{2}; 3\right]}.
                \]

                \part $(2x+3)(2-x)+(2-x)(3x-1) \geq 0$.
                \begin{align*}
                    \iff & (2-x)\left[(2x+3) +(3x-1)\right] \geq 0 \\
                    \iff & (2-x)(2x+3+3x-1)\geq 0                  \\
                    \iff & (2-x)(5x+2)\geq 0.
                \end{align*}\\
                \begin{tikzpicture}
                    \tkzTabInit[lgt=4]{$x$/1,
                        $2-x$/1,
                        $5x+2$/1,
                        $(2-x)(5x+2)$/1}{$-\infty$,$ -\dfrac{2}{5} $,$2$,$+\infty$}
                    \tkzTabLine{,+,t,+,z,-,}
                    \tkzTabLine{,-,z,+,t,+,}
                    \tkzTabLine{,-,z,+,z,-,}
                \end{tikzpicture}\\
                \[
                    \boxed{S=\left[-\dfrac{2}{5};2\right]}.
                \]

                \part $(2x-1)(2-x)-\left(\dfrac{x}{2}+1\right)(5-4x) \leq 0$.
                \begin{align*}
                    \iff & 4x- 2x^2 - 2 + x - \left( \dfrac{5x}{2} - 2x^2 + 5 - 4x\right) \leq 0 \\
                    \iff & 5x - 2x^2 - 2 - \dfrac{5x}{2} + 2x^2 - 5 + 4x \leq 0                  \\
                    \iff & \dfrac{13x}{2} - 7 \leq 0                                             \\
                    \iff & \dfrac{13x}{2}\leq 7                                                  \\
                    \iff & x\leq 7\times\dfrac{2}{13}                                            \\
                    \iff & x\leq \dfrac{14}{13}.
                \end{align*}
                \begin{tikzpicture}
                    \tkzTabInit[lgt=7]{$x$/1,
                        $(2x-1)(2-x)-\left(\dfrac{x}{2}+1\right)(5-4x)$/1}{$-\infty$,$\dfrac{14}{13}$,$+\infty$}
                    \tkzTabLine{,-,z,+,}
                \end{tikzpicture}\\
                Alors,
                \[
                    \boxed{S=\left]-\infty;\dfrac{14}{13}\right]}
                \]

                \part $(x-1)^2 - (2x+1)^2\geq 0$
            \begin{align*}
                \iff & \left(x-1-\left(2x+1\right)\right)\left(x-1+\left(2x+1\right)\right) \geq 0 \\
                \iff & \left(-x-2\right)\left(3x\right)\geq 0.
            \end{align*}
            \begin{align*}
                 & -x-2 \geq 0 \iff -2\geq x \iff x\leq -2. \\
                 & 3x \geq 0 \iff x\geq 0.
            \end{align*}
            Ainsi : \\
            \begin{tikzpicture}
                \tkzTabInit[lgt=4]{$x$/1,
                    $-x-2$/1,
                    $3x$/1,
                    $(x-1)^2 - (2x+1)^2$/1}{$-\infty$,$-2$,$0$,$+\infty$}
                \tkzTabLine{,+,z,-,t,-,}
                \tkzTabLine{,-,t,-,z,+,}
                \tkzTabLine{,-,z,+,z,-,}
            \end{tikzpicture}\\
            \[
                \boxed{S= [-2;0]}.
            \]

        \end{parts}
    \end{solution}


    \question Résoudre l'inéquation dans $\mathbb{R}$. Donner la solution sous forme d'ensemble.
    \[ \dfrac{x^3-x^2+2x-2}{(x^2 - \sqrt{2}x+1)(x^2+\sqrt{2}x+1) } < 0.\]

    \begin{solution}

        Tout d'abord étudions le dénominateur. Notons $P(x)=(x^2-\sqrt{2}x +1)(x^2+\sqrt{2}x+1)$.\\
        A priori, on ne connaît pas encore le signe de $P$. Essayons de développer afin de voir si l'on peut se débloquer.
        \begin{align*}
            P(x) & =(x^2-\sqrt{2}x +1)(x^2+\sqrt{2}x+1)                                                                                             \\
                 & = x^2(x^2 + \sqrt{2}x+1) -\sqrt{2}x(x^2+\sqrt{2}x+1) + 1\times(x^2 +\sqrt{2}x+1)                                                 \\
                 & = x^4 + \sqrt{2}x^3 + x^2 - \sqrt{2}x^3 - \sqrt{2}^2x^2 - \sqrt{2}x + x^2 + \sqrt{2}x +1                                         \\
                 & = x^4 + \cancel{\sqrt{2}x^3} + \cancel{2x^2} - \cancel{\sqrt{2}x^3} - \cancel{2x^2} - \cancel{\sqrt{2}x} + \cancel{\sqrt{2}x} +1 \\
                 & = x^4 +1 > 0.
        \end{align*}
        Par conséquent, le signe du dénominateur est strictement positif quel que soit $x$. Cela nous enlève déjà une épine du pied !\\
        \'Etudions le numérateur que nous notons $Q(x)$, et tentons de factoriser. On peut faire apparaître un facteur commun, le terme $(x-1)$ :
        \begin{align*}
            Q(x) & = x^3-x^2 + 2x - 2  \\
                 & = x^2(x-1) + 2(x-1) \\
                 & = (x-1)(x^2 + 2).
        \end{align*}
        \begin{align*}
             & x-1 \geq 0 \iff x\geq 1.       \\
             & x^2+2 > 0 \iff x\in\mathbb{R}.
        \end{align*}
        On peut donc affirmer que le signe de $Q(x)$ est le même que celui de $(x-1)$ sur $\mathbb{R}$. Par conséquent, $\dfrac{Q(x)}{P(x)}$ est négatif sur $]-\infty;1]$ et positif
        sur $[1;+\infty[$. Le rapport $\dfrac{Q(x)}{P(x)}$ s'annulle en $x=1$, donc sera strictement négatif sur $]-\infty ; 1[$.
        La solution à l'inéquation est donc
        \[
            \boxed{S=]-\infty ; 1[}.
        \]
    \end{solution}

    \question On considère la fonction $f$ définie sur $\mathbb{R}$ par la relation :
    \[ f(x)= x^3 - 3x +2. \]
    \begin{parts}
        \part \'Etablir l'égalité suivante : $f(x)=(x+2)(x-1)^2.$
        \part Dresser le tableau de signes de la fonction $f$ sur $\mathbb{R}$.
        \part En déduire l'ensemble des solutions de l'inéquation $f(x)>0$.
    \end{parts}

    \begin{solution}
        \begin{parts}
            \part Il suffit de développer l'expression donnée :
            \begin{align*}
                (x+2)(x-1)^2 & = (x+2)(x^2-2x+1)                \\
                             & = x^3 - 2x^2 + x + 2x^2 - 4x + 2 \\
                             & = x^3 - 3x + 2
                             & = f(x).
            \end{align*}

            \part On peut maintenant dresser le tableau de signes de $f$ sur $\mathbb{R}$.\\
            \begin{tikzpicture}
                \tkzTabInit[lgt=4]{$x$/1,
                    $x+2$/1,
                    $(x-1)^2$/1,
                    $(x+2)(x-1)^2$/1}{$-\infty$, $-2$,$ 1 $, $+\infty $}
                \tkzTabLine{,-,z,+,t,+,}
                \tkzTabLine{,+,t,+,z,+,}
                \tkzTabLine{,-,z,+,z,+,}
            \end{tikzpicture}\\

            \part On en déduit donc que l'ensemble solution de l'inéquation $f(x) > 0$ est
            \[
                \boxed{S=]-\infty ; -2[}.
            \]

        \end{parts}
    \end{solution}



    \question (Croissance et décroissance)
    \begin{parts}
        \part Démontrer que la fonction définie par $f(x)=3x+2$ est croissante sur $\mathbb{R}$.
        \part Démontrer que la fonction définie par $f(x)=2-x$ est décroissante sur $\mathbb{R}$.
        \part Démontrer que la fonction définie par $f(x)=2x^2-2x+1$ est décroissante sur $\left]-\infty ; \dfrac{1}{2}\right[$.
        \part Démontrer que la fonction définie par $f(x)=-2x^2+x^4$ est décroissante sur $\left[0;1\right]$ et croissante sur $\left[1;+\infty\right[$.
    \end{parts}

    \begin{solution}
        \begin{parts}
            \part Soit $u,v\in \mathbb{R}$ tels que $u \leq v$. \'Etudions le signe de $f(v)-f(u)$ :
            \begin{align*}
                f(v)-f(u) & = 3v +2 - (3u+2) \\
                          & = 3v +2 - 3u -2  \\
                          & = 3(v-u).
            \end{align*}
            Comme par hypothèse $u \leq v$, alors $v-u \geq  0$. Par conséquent $3(v-u)\geq 0$. Donc $f(v)-f(u)\geq 0$, soit donc $f(v) \geq f(u)$. Ainsi, $f$ est donc croissante sur $\mathbb{R}$.

            \part Soit $u,v \in \mathbb{R}$ tels que $u \leq v$. \'Etudions le signe de $f(v)-f(u)$ :
            \begin{align*}
                f(v)-f(u) & = 2-v - (2-u) \\
                          & = 2-v - 2 + u \\
                          & = u-v.
            \end{align*}
            Par hypothèse, $u \leq v$, donc $u-v\leq 0$. D'où, $f(v)-f(u) \leq 0$, soit donc $f(u) \geq f(v)$. Donc $f$ est décroissante sur $\mathbb{R}$.

            \part Soit $u,v \in ]-\infty; \frac{1}{2}[$ tels que $u \leq v$. \'Etudions le signe de $f(u)-f(v)$ :
                        \begin{align*}
                            f(u)-f(v) & = 2u^2 - 2u +1 - 2v^2 + 2v - 1 \\
                                      & = 2(u^2 - v^2) - 2(u-v)        \\
                                      & = 2(u-v)(u+v) - 2(u-v)         \\
                                      & = 2(u-v)(u+v-1)                \\
                        \end{align*}
                        Sachant que $u \leq v$, alors $u - v\leq 0$.\\
                        De plus sachant que $u,v\in ]-\infty;\frac{1}{2}]$, soit donc $u \leq \frac{1}{2}$, et $v\leq \frac{1}{2}$. Par conséquent $u+v\leq 1$. Ainsi, $u+v-1\leq 0$. \\
            Le produit des termes $(u-v)$ et $(u+v-1)$ est donc positif (car produit de deux termes de même signe). Ainsi, $f(u)-f(v)\geq 0$, et donc $f(u)\geq f(v)$. Alors f est décroissante sur $]-\infty; \frac{1}{2}[$.

            \part \begin{subparts} \subpart Soit $u,v\in [0,1]$ tels que $u \leq v$. Montrons que $f(u)-f(v) \geq 0$.\\
                \begin{align*}
                    f(u)-f(v) & = -2u^2 + u^4 - ( -2v^2 + v^4 )         \\
                              & = -2u^2 + u^4 + 2v^2 - v^4              \\
                              & = -2(u-v)(u+v) + (u^2)^2 - (v^2)^2      \\
                              & = -2(u-v)(u+v) + (u^2 - v^2)(u^2 + v^2) \\
                              & = -2(u-v)(u+v) + (u-v)(u+v)(u^2 + v^2)  \\
                              & = (u-v)(u+v)(-2+ (u^2 + v^2))           \\
                \end{align*}
                Sachant que $u\leq v$, alors $u-v\leq0$. Donc $u-v$ est négatif.\\
                Reste à déterminer le signe de l'autre terme :
                \begin{align*}
                     & 0 \leq u \leq 1 \\
                     & 0 \leq v\leq 1  \\
                     & 0\leq u+v\leq 2 \\
                \end{align*}
                Donc le terme $u+v\geq 0$ de toutes façons.\\
                De plus,
                \begin{align*}
                                & 0\leq u \leq 1                                                                    \\
                                & 0 \leq u^2 \leq 1 (\text{car la fonction carrée est croissate sur $[0;+\infty[$}) \\
                                & 0 \leq v \leq 1                                                                   \\
                                & 0 \leq v^2 \leq 1 (\text{car la fonction carrée est croissate sur $[0;+\infty[$}) \\
                    \Rightarrow & 0 \leq u^2 + v^2 \leq 2
                \end{align*}
                Ainsi, $u^2+v^2 - 2 \leq 0$. Par conséquent, le produit $(u-v)(u+v)(-2+ u^2 + v^2) \geq 0$, car $(u-v)\leq 0$, $(u+v)\geq 0$, et $(-2 + (u^2 +v^2)) \leq 0$.\\
                Ainsi, $f(u)-f(v) \geq 0$, donc $f(u)\geq f(v)$. Donc $f$ est décroissante sur $[0;1]$.
                \subpart Soit $u,v \in [1;+\infty[$ tels que $u \leq v$. Montrons que $f(u) \leq f(v)$.\\
                \begin{align*}
                    f(u)-f(v) & = (u-v)(u+v)(-2+u^2+v^2).
                \end{align*}
                Le terme $(u-v)$ reste négatif car $u\leq v$. De même pour $u+v$ qui reste positif.\\
                On a par hypothèse :
                \begin{align*}
                                & u \geq 1                                                                   \\
                                & v \geq 1                                                                   \\
                                & u^2 \geq 1 (\text{car la fonction carrée est croissate sur $[0;+\infty[$}) \\
                                & v^2\geq 1 (\text{car la fonction carrée est croissate sur $[0;+\infty[$})  \\
                    \Rightarrow & u^2 + v^2 \geq 2.
                \end{align*}
                Par conséquent, $u^2+v^2 -2 \geq 0$. Ainsi, la différence $f(u)-f(v)$ est du signe de $(u-v)$, qui est négatif.
                Donc finalement, $f(u)\leq f(v)$, et donc $f$ est croissante sur $[1;+\infty[$.


            \end{subparts}


        \end{parts}

    \end{solution}




    \question (Probabilités d'événements)
    On considère l'expérience suivante : on lance simultanément deux dés équilibrés \underline{\textbf{numérotés de 5 à 10}}.
    \begin{parts}
        \part Décrire rapidement l'univers de l'expérience, que l'on notera $\Omega$.
        \part Justifier que les issues obtenues sont équiprobables.
        \part Réaliser les tableaux à double entrée suivants :
        \begin{enumerate}
            \item L'un représentant celui de la somme des deux dés\footnote{La somme de deux nombres est l'addition des deux nombres.}.
            \item L'autre représentant celui du produit des deux dés\footnote{Le produit de deux nombres est la multiplication des deux nombres.}.
        \end{enumerate}
        \part On note les événements suivants :
        \begin{itemize}
            \item[] A : \og Obtenir que la somme des deux nombres obtenus soit égal à 7 \fg.
            \item[] B : \og Obtenir que la somme des deux nombres obtenus soit égale à 15 \fg.
            \item[] C : \og Obtenir que le produit des deux nombres obtenus soit inférieur ou égal à 45 \fg.
            \item[] D : \og Obtenir que le produit des deux nombres obtenus soit supérieur à 65 \fg.
            \item[] E : \og Obtenir que la somme des deux nombres soit supérieure ou égale à 12, \textbf{et} dont le produit est inférieur ou égal à 40 \fg.
        \end{itemize}
        \underline{Question} : Calculer les probabilités des événements $A$, $B$, $C$, $D$ et $E$.
    \end{parts}

    \begin{solution}
        \begin{parts}
            \part L'univers de l'expérience, $\Omega$ est défini par l'ensemble des couples que l'on peut former avec les chiffres de 5 à 10 :
            \[
                \Omega =  \left\lbrace 5;6;7;8;9;10\right\rbrace ^2=\left\lbrace (5;5),(5;6),\dots,(5;10),(6;5),(6;6),\dots,(10;5),\dots,(10;10)\right\rbrace.
            \]

            \part Les issues sont équiprobabiles car les deux dés sont \emph{équilibrés}. Chacune des issues a la même probabilité d'être tirée.

            \part \begin{multicols}{2} \begin{tabular}{|l||l|l|l|l|l|l|}
                    \hline
                    $+$ & 5  & 6  & 7  & 8  & 9  & 10 \\ \hline\hline
                    5   & 10 & 11 & 12 & 13 & 14 & 15 \\ \hline
                    6   & 11 & 12 & 13 & 14 & 15 & 16 \\ \hline
                    7   & 12 & 13 & 14 & 15 & 16 & 17 \\ \hline
                    8   & 13 & 14 & 15 & 16 & 17 & 18 \\ \hline
                    9   & 14 & 15 & 16 & 17 & 18 & 19 \\ \hline
                    10  & 15 & 16 & 17 & 18 & 19 & 20 \\ \hline
                \end{tabular}

                \begin{tabular}{|l||l|l|l|l|l|l|}
                    \hline
                    $\times$ & 5  & 6  & 7  & 8  & 9  & 10  \\ \hline\hline
                    5        & 25 & 30 & 35 & 40 & 45 & 50  \\ \hline
                    6        & 30 & 36 & 42 & 48 & 54 & 60  \\ \hline
                    7        & 35 & 42 & 49 & 56 & 63 & 70  \\ \hline
                    8        & 40 & 48 & 56 & 64 & 72 & 80  \\ \hline
                    9        & 45 & 54 & 63 & 72 & 81 & 90  \\ \hline
                    10       & 50 & 60 & 70 & 80 & 90 & 100 \\ \hline
                \end{tabular}

            \end{multicols}
            \part \begin{enumerate} \item L'événement $A$ est impossible. Le plus petit nombre que l'on puisse obtenir est 10. Donc $\mathbb{P}(A)=0$.

                \item $B$ est donné par la deuxième diagonale, où tous les nombres valent 15. Il y a 6 de ces nombres sur 36 au total, donc $P(B)=\dfrac{\text{nombre de cas favorables}}{\text{nombre de cas total}}=\dfrac{6}{36}=\dfrac{1}{6}$.
                \item On regarde maintenant le second tableau. On dénombre 12 cases dont la valeur est inférieure ou égale à 45. Alors, $P(C)=\dfrac{12}{36}=\dfrac{1}{3}$.

                \item On dénombre 10 valeurs du tableau qui sont supérieures à 65. Ainsi, $P(D)=\dfrac{10}{36}=\dfrac{5}{18}$.
                \item Là, il faut faire attention. Il faut que le produit soit inférieur ou égal à 40. Par conséquent, on cherchera parmis ces couples lesquels ont une somme supérieure à 12. Les couples dont le produit est inférieur à 40 : \[(5;5),(6;5),(7;5),(8;5),(5;6),(6;6),(5;7),(5;8)\]. Parmi cette liste, les seuls couples dont la somme est supérieure ou égale à 12 sont : $(7;5),(8;5),(6;6),(5;7),(5;8)$. Par conséquent, $P(E)=\dfrac{5}{36}$.

            \end{enumerate}



            % Please add the following required packages to your document preamble:
            % \usepackage{graphicx}


        \end{parts}

    \end{solution}


    \question (Bonus) Jean dispose d'un gâteau d'anniversaire circulaire et uniforme (pas de glaçage ni de décoration au dessus). Il a invité 8 amis à son anniversaire. \\
    \underline{Question} : Comment découper le gâteau en 8 parts \textbf{équitables} en seulement \textbf{3 tranchers de couteau} ?

    \begin{solution}
        Il fallait simplement couper en 4 parts égales simples, puis réaliser une découpe sur \emph{l'épaisseur}, ce qui donne donc 8 parts.
    \end{solution}

\end{questions}

\end{document}


